\documentclass{article}
\usepackage{fontspec,xunicode,polyglossia}		% Support unicode
\setmainfont{Linux Libertine O}					% Police partout, justice nul part
\setmainlanguage{french}						% Langue de molière
\usepackage[a4paper]{geometry}					% Taille du papier
\usepackage{csquotes}							% Pour gérer finement les citations
\usepackage{hyperref}							% Liens dans le PDF
% Bibliographie
\usepackage[citestyle=verbose]{biblatex}
\bibliography{comparatif-logiciels}
\renewcommand{\newunitpunct}[0]{\addcomma\addspace}
% En tête et pied de page
\usepackage{fancyhdr}							
\pagestyle{fancy}
\renewcommand{\headrulewidth}{0pt}
\renewcommand{\footrulewidth}{0.4pt}
\lhead{Maïeul Rouquette}
\rhead{Logiciels d'ecdotique}
\chead{}
\lfoot{\href{https://creativecommons.org/licenses/by-sa/3.0/fr/}{Licence CC Fr 3.0 Paternité - Partage à l'Identique}}
\rfoot{\thepage}
\cfoot{}
\begin{document}
\author{Maïeul Rouquette}
\title{Comparatif succinct de quelques outils informatiques pour l'édition critique\footnote{Sources disponibles sur \url{http://github.com/maieul/logiciels-ecdotique}.}}

\maketitle\thispagestyle{fancy}
\tableofcontents
\section{Pour les éditions papiers}
\subsection{Word ou LibreOffice}

\subsubsection{Avantages}
\begin{itemize}
	\item Rien à apprendre.
	\item Très répandu.
	\item Utilisable en lien avec Zotero pour la bibliographie.
\end{itemize}

\subsubsection{Inconvénients}
\begin{itemize}
	\item Pas du tout adapté pour produire des notes critiques.
	\item Pas du tout adapté pour gérer des éditions en parallèle.
	\item Typographie de médiocre qualité.
\end{itemize}

\subsection{Logiciel de PAO (Indesign, Scribus etc.)}

\subsubsection{Avantages}
\begin{itemize}
	\item Permet de gérer finement la typographie.
\end{itemize}

\subsubsection{Inconvénients}
\begin{itemize}
	\item On doit tout faire \enquote{à la main} : pas d'outil pour gérer les éditions en parallèle ou la numérotation automatique des lignes associées aux notes.
	\item Les logiciels propriétaires sont très chers.
	\item Souvent formats propriétaires.
\end{itemize}

Logiciels plus pour l'éditeur que pour l'auteur.

\subsection{Classical Text Editor}
\subsubsection{Avantages}
\begin{itemize}
	\item Prévu spécifiquement pour les éditions critiques et les notes en parallèles.
	\item Fréquemment mis à jour.
	\item Outils d'export vers divers formats.
\end{itemize}

\subsubsection{Inconvénients}
\begin{itemize}
	\item Pas de module de gestion bibliographique.
	\item Format propriétaire gênant la pérennité des données dans le temps.
	\item Assez cher.
\end{itemize}

\subsection{\LaTeX{} + eledmac/eledpar}
\subsubsection{Avantages}
\begin{itemize}
	\item Logiciel libre et gratuit.
	\item Format ouvert.
	\item Format \enquote{texte formaté} permettant d'utiliser des outils de suivi des versions et de travail collaboratif.
	\item Outre l'édition critique et les textes en parallèle, gère notamment :
		\begin{itemize}
			\item Index, dont biblique.
			\item Bibliographie, avec ou sans indexation des références.
			\item Dessin vectoriel, permettant de tracer des stemma.
		\end{itemize}
	\item Deux livres récents sur le sujet\footnote{C'est deux livres parlent de ledmac et non de eledmac. Le débutant commencera directement avec eledmac.} :
		\begin{itemize}
		\item \cite[disponible également en libre téléchargement]{Rouquette2012}.
		\item \cite{leal2012}.
		\end{itemize}
\end{itemize}

\subsubsection{Inconvénients}
	\begin{itemize}
		\item Nécessite un petit temps de prise en main.
		\item Quelques cas typographiques non gérables.
		\item Export vers d'autres formats que PDF assez difficile.
	\end{itemize}
\section{Pour le WEB}
\subsection{Principe}
On encodera les données selon la \enquote{norme} XML-TEI à l'aide d'un éditeur de texte, éventuellement spécialisé. Puis on transformera ces données en pages web grâce à une feuille de style XLST.

\subsection{Avantages}
\begin{itemize}
	\item Format libre et ouvert, normé.
	\item Format \enquote{texte formaté} permettant d'utiliser des outils de suivi des versions et de travail collaboratif.
	\item Possibilité d'encoder de nombreuses données.
\end{itemize}

\subsection{Inconvénients}
\begin{itemize}
	\item Vocabulaire très riche, donc un certain temps est nécessaire pour l'acquérir.
	\item Comme on peut encoder beaucoup plus de détails que ce qu'on mettrait sur une version papier, on prend plus de temps.
	\item Il faut quasiment prévoir une feuille XLST par projet. Si le XML-TEI est assez simple à apprendre, le XLST est plus complexe. Peu adapté pour les doctorants n'ayant pas l'âme d'un codeur et n'étant pas dans un projet collectif.
\end{itemize}
\end{document}